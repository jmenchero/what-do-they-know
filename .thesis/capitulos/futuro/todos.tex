
Durante la realización de este proyecto se han identificado una serie de posibles ampliaciones y mejoras que se podrán incluir en futuras versiones del mismo. Dichas mejoras se detallan a continuación:

\begin{enumerate}
    \item Mejorar las politicas de acceso IAM.
    \item Migrar claves de acceso a GitHub secrets.
    \item Contratar a un UX Designer para replantear el aspecto gráfico de la plataforma.
    \item Comprar un dominio y liberar el acceso a la aplicación.
    \item Promocionar entre la comunidad de desarrolladores.
    \item Tests unitarios para garantizar la calidad de los reportes tras posibles cambios.
    \item Tests end to end para garantizar la funcionalidad básica de la plataforma en cada despliegue.
    \item Asegurar funcionamiento con diferentes versiones de los datos exportados.
    \item RF-04 [Procesar datos de What's App]: Permitir subir el histórico de mensajes de What's App.
    \item RF-08 [Intereses]: Resumir las temáticas principales de las conversaciones del usuario.
    \item RF-09 [Interacciones]: Mostrar las personas que han aparecido o desaparecido de la vida del usuario a partir de su frecuencia de mensajes.
    \item RF-10 [Estado anímico]: Simular una progresión anímica basándose en los emojis más utilizados según el periodo.
    \item RF-11 [Contactos favoritos]: Mostrar a qué contactos responde más rápido el usuario.
    \item RF-12 [Contactos más sociales]: Visualizar un ranking con las personas en las que el usuario comparta más grupos.
\end{enumerate}
