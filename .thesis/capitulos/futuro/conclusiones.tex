
Los objetivos principales de este proyecto eran concienciar al público de la información que los actuales propietarios de sus datos pueden obtener sobre su intimidad, entretener a los visitantes del sitio con los infográficos generados y sentar las bases para una comunidad de desarrollo de código abierto.

Esto se ha conseguido mediante la implementación de una herramienta web accesible a través de internet\cite{Plataforma} en la que se ha guiado a los usuarios en el proceso de obtención de sus datos y dado acceso a un set de infográficos con información elaboarada a partir de sus históricos de aplicaciones de mensajería instantánea. A su vez la implementación modular y de código abierto, accesible a través de GitHub\cite{Repositorio}, espera generar una comunidad colaborativa una vez se publicite entre la comunidad de desarrolladores.

A su vez, este proyecto me ha permitido investigar y aplicar buenas prácticas del desarrollo y arquitectura de software, en un entorno en producción real y accesible por el público, así como trabajar en un proyecto con un proceso de integración continua definido desde el principio y con tecnologías de cloud computing como las provistas por Amazon AWS, pudiendo hacerme cargo de todas las fases de un proyecto público, desde su conceptualización, diseño de experiencia de usuario, arquitectura e implementación hasta su publicación.
