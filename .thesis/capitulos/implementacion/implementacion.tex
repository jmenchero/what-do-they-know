
Tras lo planteado en el capítulo 3, la selección para el framework de frontend es Vue.js, por su crecimiento de uso en la comunidad así como por su curva de aprendizaje.

Vue.js es un framework que extiende HTML, JavaScript y CSS ofreciendo una sintaxis propia enriquecida que permite la separación conceptual del código en componentes reutilizables (siguiendo los principios de reusabilidad de componentes gráficos descritos en el libro Atomic Design\cite{AtomicDesign}) esperando así facilitar la contribución de código por parte de la comunidad (RNF-10 [Comunidad de desarrollo]),  y un sistema de reactividad integrado que permite regenerar nodos HTML ante cambios de los datos almacenados, facilitando una carga dinámica del contenido. También permite introducir conceptos de programación estructurada dentro del paradigma declaritivo de HTML, siendo más realista con la integración actual con JavaScript del lenguaje de marcado.

El framework Nuxt.js también ha sido utilizado en la aplicación, por ser una capa por encima de Vue.js, Node.js y Webpack orientada a facilitar el enrutado entre URLs y componentes, dejando una estructura más intuitiva de la aplicación separando los conceptos de páginas y componentes, y permitiéndonos generar un build de la web de manera estática con el modo Single Page Application.

Vuex

Yarn

Buefy
