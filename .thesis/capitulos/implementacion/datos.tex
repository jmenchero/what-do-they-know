
Una vez cargado el input del usuario en memoria (el histórico de mensajes) comienza el procesado de los datos con el objetivo de elaborar resúmenes de los mismos, simples e independientes, con un formato fácilmente legible por las vistas que se encargarán de mostrar sus resultados de manera visual.

Para ello existen dos módulos principales en la aplicación, por un lado encontramos la conocida como store utilizando la librería Vuex, una implementación para Vue del State Management Pattern\cite{VueJS}. Este patrón consiste en la aplicación de un flujo de una sola dirección en los datos. De forma que las acciones mutan el estado de los datos, y este es representado en la vista, pero manteniendo dicho estado en un singleton común a toda la aplicación para permitir su consulta por parte de toda la aplicación, asegurando que solo se consultan los datos en modo lectura, y que cualquier mutación se canaliza a través de un solo punto como se detalla en la figura \ref{FIG:VUEX}.

\begin{figure}[Diagrama de flujo de Vuex]{FIG:VUEX}{Diagrama de flujo de Vuex. Fuente: VueJS.org}
    \image{12cm}{}{vuex}
\end{figure}

En este módulo se encuentran por un lado las acciones encargadas de procesar los datos generando los reportes, y por otro el estado de la aplicación, con la información de los reportes almacenada y accesible por cualquier componente visual de la misma.

El otro módulo es el encargado de realizar las conexiones entre el cliente y la API de datos de Redshift, tanto para volcar los datos de aquellos usuarios que acepten compartirlos como para realizar las consultas analíticas correspondientes para obtener los reportes globales.
