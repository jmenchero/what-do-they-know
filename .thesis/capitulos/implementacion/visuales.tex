
Tras lo planteado en la sección \ref{SEC:FRONTEND}, la selección final para el framework de frontend es Vue.JS, por su crecimiento de uso en la comunidad así como por su asequible curva de aprendizaje.

\begin{figure}[Division por componentes]{FIG:COMPONENTES}{Diagrama de asociación entre organización visual y estructura de ficheros. Fuente: VueJS.org}
    \image{12cm}{}{componentes}
\end{figure}

Vue.JS es un framework que extiende HTML, JavaScript y CSS\cite{VueJS}. Estos lenguajes se integran bajo una sintaxis propia enriquecida que permite la separación conceptual del código en componentes con estructura, funcionalidad y estética reutilizables, atendiendo a una asociación entre componentes gráficos y ficheros como se observa en la figura \ref{FIG:COMPONENTES}. Se siguen así los principios de reusabilidad de componentes gráficos descritos en el libro Atomic Design\cite{AtomicDesign}. Gracias a esta modularización se espera facilitar la contribución de código por parte de la comunidad (RNF-10 [Comunidad de desarrollo]).

\begin{figure}[Ejemplo de componente en Vue]{FIG:VUE}{Ejemplo de componente de la plataforma en Vue}
    \image{9cm}{}{vue}
\end{figure}

En la figura \ref{FIG:VUE} se describe un componente real de la aplicación de tipo botón. Donde se puede obsverar cómo el componente es definido por una plantilla de la estructura HTML con contenido inyectable (en este caso el texto a mostrar en el botón), seguido de la funcionalidad JavaScript capaz de manejar los datos a los que tiene acceso el ámbito del componente y emitir eventos a su padre, y finalizando con las clases CSS que definen la especificación estética del componente. 

El framework ofrece a su vez un sistema de reactividad integrado que permite regenerar nodos HTML ante cambios de los datos almacenados, facilitando una carga dinámica del contenido. También permite introducir conceptos de programación estructurada dentro del paradigma declarativo de HTML, siendo más realista con la integración actual entre JavaScript y el lenguaje de marcado. 

El framework Nuxt.JS también ha sido utilizado en la aplicación, por ser una capa por encima de Vue.JS, Node.JS y Webpack. Este framework introduce a su vez el concepto de páginas, que no son más que componentes de Vue para los que automáticamente se genera una ruta de acceso web \cite{NuxtJS}. Esto permite omitir el paso de configuración de rutas resultando en una estructura más intuitiva de la aplicación separando los conceptos de páginas y componentes.

A su vez, la ausencia de un generador de rutas dinámico permite generar una distribución estática de la web (gracias al modo de compilación Single Page Application). Se puede así prescindir de un servidor web dinámico, permitiéndonos el uso de AWS S3 como hosting estático y AWS CloudFront como CDN (en el caso de no compilar un contenido estático, no podríamos hacer uso de un CDN para garantizar el acceso global a la aplicación, sino que deberíamos proveer de réplicas del servidor web sincronizadas a lo largo del globo).

También se ha empleado la librería Fullpage.JS, que permite la navegación en formato carrusel\cite{FullpageJS}, similar a la de la iniciativa de referencia Spotify Wrapped. El conjunto de estas tres tecnologías, Vue.JS para los componentes, Nuxt.JS para las páginas y Fullpage.JS para la navegación en formato carrusel, ha permitido asociar la división conceptual de las dos principales secciones de la experiencia de usuario (introducción y análisis) en dos ficheros de páginas diferentes, y cada pantalla o reporte en su propio fichero de componente aislado como se aprecia en la figura \ref{FIG:VISUALFILES}.

\begin{figure}[Estructura de ficheros]{FIG:VISUALFILES}{Estructura de ficheros de los componentes visuales de la plataforma}
    \image{7cm}{}{visualfiles}
\end{figure}

Cada componente de la lista de reportes se encarga de describir la organización de los elementos visuales que contiene su reporte, sus características estéticas y de transformar los datos, ya preparados y listos para consumir, al infográfico correspondiente.

Para ciertos elementos gráficos se emplea la librería Buefy, que provee de una serie de componentes responsive atómicos (como botones, slides, ...) listos para ser empleados en el sistema de datos reactivos de Vue\cite{Buefy}.
