
\begin{itemize}
    \item \textbf{RF-01 [Presentación del proyecto]}: Presentar el proyecto de manera completa y breve, para evitar perder al usuario antes de conocer de qué trata.
    \item \textbf{RF-02 [Guía de exportación]}: Guiar al usuario durante el proceso de exportación de sus datos de la aplicación de mensajería externa seleccionada.
    \item \textbf{RF-03 [Procesar datos de Telegram]}: Permitir subir el histórico de mensajes de Telegram.
    \item \textbf{RF-04 [Procesar datos de What´s App]}: Permitir subir el histórico de mensajes de What´s App.
    \item \textbf{RF-05 [Emojis más usados]}: Mostrar una lista con los emojis más usados.
    \item \textbf{RF-06 [Uso diario]}: Desglosar la actividad diaria de la aplicación.
    \item \textbf{RF-07 [Horas de sueño o trabajo]}: Mostrar durante qué horas se puede deducir que el usuario duerme o trabaja.
    \item \textbf{RF-08 [Intereses]}: Resumir las temáticas principales de las conversaciones del usuario.
    \item \textbf{RF-09 [Interacciones]}: Mostrar las personas que han aparecido o desaparecido de la vida del usuario a partir de su frecuencia de mensajes.
    \item \textbf{RF-10 [Estado anímico]}: Simular una progresión anímica basándose en los emojis más utilizados según el periodo.
    \item \textbf{RF-11 [Contactos favoritos]}: Mostrar a qué contactos responde más rápido el usuario.
    \item \textbf{RF-12 [Contactos más sociales]}: Visualizar un ranking con las personas en las que el usuario comparta más grupos.
\end{itemize}
