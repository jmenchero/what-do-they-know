
La computación en la nube\cite{ArchitectingCloud} es la oferta de servicios de computación tales como servidores, bases de datos, redes privadas, software, herramientas de análisis, ... a través de internet, sin la necesidad de disponer de hardware físico en propiedad y beneficiándose de la economía de escala que aporta el compartir recursos de forma flexible entre un gran número de usuarios.

Actualmente hay numerosos proveedores de cloud, a continuación se esbozan las principales ventajas de aquellos con más cuota de mercado según el gráfico de la figura \ref{FIG:CLOUD}:

\begin{itemize}
    \item \textbf{Amazon Web Services}: Mayor presencia en el mercado, gran número de servicios, disponibilidad global con escalabilidad y replica entre regiones, 99.99\% de disponibilidad garantizada\cite{AWS}.
    \item \textbf{Microsoft Azure}: Destacan sus servicios de aprendizaje automático e inteligencia artificial\cite{Azure}.
    \item \textbf{Google Cloud}: Como creadores de Kubernetes, es la solución mejor integrada para la utilización de contenedores de Docker en organizaciones con arquitectura de microservicios\cite{GoogleCloud}.
    \item \textbf{Digital Ocean}: Enfocados en dar servicio a desarrolladores, y no tanto a grandes organizaciones con complejos requisitos de infraestructura\cite{DigitalOcean}.
\end{itemize}

\begin{figure}[Cuota de mercado de proveedores de cloud]{FIG:CLOUD}{Histórico de la progresión de cuota de mercado de proveedores de cloud. Fuente: Synergy Research Group}
    \image{12cm}{}{clouds}
\end{figure}

\newpage
