El 14 de abril de 2016 se aprobo en el Parlamento Europeo el Reglamento General de Proteccion de Datos, entrando en vigor el 24 de Mayo de 2016 y concediendo un periodo de aplicacion de dos anyos hasta el 24 de Mayo. A partir del 25 de Mayo de 2018 todas las empresas, organizaciones, organismos o instituciones dentro del marco europeo comenzaron a tener la obligacion, bajo multa por incumplimiento de hasta 20 millones de euros, de ofrecer a los usuarios de una manera facilmente accesible y legible una copia de sus datos almacenados.

Gracias a esta ley podemos solicitar ante cualquier plataforma que almacene o trate nuestros datos, una copia de seguridad de toda nuestra informacion que tienen disponible, desde Google, Facebook, Twitter, Spotify, Tinder, o, en el caso de la informacion a analizar en este trabajo, a aplicaciones de mensajeria instantanea como What's App o Telegram. Permitiendonos tener acceso a una fuente de datos de manera sencilla y facilmente ingerible por el sistema propuesto.
