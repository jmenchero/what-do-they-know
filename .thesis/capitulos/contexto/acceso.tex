
El 14 de abril de 2016 se aprobó en el Parlamento Europeo el Reglamento General de Protección de Datos\cite{RGPD}, entrando en vigor el 24 de Mayo de 2016 y concediendo un periodo de aplicación de dos años hasta el 24 de Mayo de 2018. A partir del 25 de Mayo de 2018 todas las empresas, organizaciones, organismos o instituciones dentro del marco europeo comenzaron a tener la obligación, bajo multa por incumplimiento de hasta 20 millones de euros, de ofrecer a los usuarios de una manera fácilmente accesible y legible una copia de sus datos almacenados.

Gracias a esta ley podemos solicitar ante cualquier plataforma que almacene o trate nuestros datos una copia de seguridad de toda nuestra información que tengan disponible. Desde Google\cite{ExportGoogle}, Facebook\cite{ExportFacebook}, Twitter\cite{ExportTwitter}, Spotify\cite{ExportSpotify}, Tinder\cite{ExportTinder} o, en el caso de la información de la que es sujeto el análisis de este trabajo, a aplicaciones de mensajería instantánea como What's App\cite{ExportWhatsApp} o Telegram\cite{ExportTelegram}.

Esto nos permite tener acceso a una fuente de datos para el proyecto de manera sencilla y fácilmente ingerible por el sistema propuesto, cuando disponer de fuentes reales de datos es usualmente la parte más complicada a la hora de aterrizar una idea conceptual en un entorno de producción.

No obstante esta ley, aunque obliga a las empresas a compartir la propiedad de los datos crudos con el individuo que los genere, no impone bajo ninguna cláusula que se compartan las conclusiones destiladas que generan a partir de esa información cruda.

\newpage
