
El espectro del frontend es muy cambiante, en apenas dos o tres años las tecnologías más presentes pueden pasar a ser consideradas obsoletas, pero gracias a iniciativas como StateOfJS, una encuesta global sobre el ecosistema front, se puede consultar de manera muy visual el estado actual de esta rama.

\begin{figure}[Frontend frameworks]{FIG:FRAMEWORKS}{Progresión de uso de frameworks de frontend. Fuente: StateOfJS.com}
    \image{12cm}{}{frameworks}
\end{figure}

Como podemos observar en el gráfico de uso, Vue.JS parece ser el único framework con un crecimiento constante y consistente en los últimos cinco años, pudiendo significar un declive del resto de competidores en los próximos años en favor de este framework, destacable por su suave curva de aprendizaje y su sintaxis de HTML enriquecida, integrando bucles, condicionales y bindings de datos en las propias etiquetas nativas de HTML.
