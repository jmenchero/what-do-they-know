
\subsection{Relacionales vs No relacionales}

\begin{itemize}
    \item \textbf{Relacionales}: Almacenan la información de manera estructurada, con una forma constante y una declaración explicita de las relaciones entre los diversos elementos (MySQL, Oracle, SQL Server, PostgreSQL, Redshift, ...).
    \item \textbf{No relacionales}: Almacenan la información como documentos independientes, que no tienen por qué mantener la misma forma ni tener relaciones explicitas (MongoDB, Redis, Elasticsearch, Cassandra, ...).
\end{itemize}

\subsection{Columnares vs Orientadas a filas}

\begin{itemize}
    \item \textbf{Columnares}: Almacenan de manera contigua cada columna o atributo, para acelerar el análisis estadístico de toda una tabla (Redshift, BigQuery, ...).
    \item \textbf{Orientadas a filas}: Almacenan en memoria contigua la información de cada entidad, permitiendo un acceso rápido a todos los datos de una misma entidad (SQLServer, PostgreSQL, ...).
\end{itemize}
