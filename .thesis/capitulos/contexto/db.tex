
Relacionales vs No Relacionales


Relacionales: Almacenan la informacion de manera estructurada, con una forma constante y una declaracion explicita de las relaciones entre los diversos elementos (MySQL, Oracle, SQL Server, PostgreSQL, ...)
No Relacionales: Almacenan la informacion como documentos independientes, que no tienen por que mantener la misma forma ni tener relaciones explicitas (MongoDB, Redis, Elasticsearch, Cassandra, ...)


Columnares vs Orientadas a filas

Orientadas a filas: Almacenan en memoria contigua la informacion de cada entidad, permitiendo un acceso rapido a todos los datos de una misma entidad (SQLServer, PostgreSQL, ...)
Columnares: Almacenan de manera contigua cada columna o atributo, para acelerar el analisis estadistico de toda una tabla (Redshift, BigQuery, ...)  
