
En esta sección analizaremos algunas aplicaciones existentes con forma u objetivos similares, tratando de observar si hay un nicho de usuarios interesados que den viabilidad al proyecto, para posteriormente enumerar las limitaciones de estos a subsanar.

\subsection{Spotify Wrapped}

Desde 2016 Spotify lanza anualmente una campaña de marketing\cite{SpotifyWrapped} que permite a sus usuarios visualizar una compilación de sus datos de uso como los mostrados en la figura \ref{FIG:SPOTIFY}, comparándolos con el resto de la comunidad que utiliza la aplicación. Desde resúmenes meramente estadísticos, así como análisis de emociones o intereses.

\begin{figure}[Spotify Wrapped]{FIG:SPOTIFY}{Ejemplos de pantallas de reporte de la iniciativa Spotify Wrapped. Fuente: Spotify.com}
    \image{15cm}{}{spotify}
\end{figure}

Esta campaña ha tenido un gran impacto social y cultural, llegando a formar parte de la cultura pop de las nuevas generaciones, y generando en el inconsciente colectivo inquietud e interés por los datos a una población regularmente ajena y desinteresada de su información en la era digital.

\newpage

\subsection{Tinder Insights}

Creada en 2019 por Dora Szucs (Software Engineer) y Krisztina Szucs (UX Designer) de manera independiente\cite{TinderInsigths}, como se puede observar en la figura \ref{FIG:TINDER} da acceso a información estadística sobre el uso de la aplicación Tinder, como la cantidad de mensajes enviados y recibidos, tiempo medio de chat, ...

\begin{figure}[Tinder Insights]{FIG:TINDER}{Ejemplos de pantallas de reporte de Tinder Insights. Fuente: TinderInsights.com}
    \image{6cm}{}{tinder1}
    \image{6cm}{}{tinder2}
\end{figure}

Siguiendo el mismo planteamiento que nuestro proyecto, requieren que el usuario solicite paralelamente desde un flujo externo a la aplicación su copia de seguridad, sirviendo como ejemplo práctico de que el interés por la información sobre uno mismo puede ser suficiente para que un gran numero de población comparta sus datos privados de uso con una aplicación de terceros para analizarla (incluyendo sus conversaciones a través de la app).

\newpage

\subsection{Chat Visualizer}

Una herramienta que permite analizar la actividad de un chat de What's App de manera estadística\cite{ChatVisualizer} (72K chats analizados en el momento de la redacción de esta memoria). Algunos ejemplos de sus infográficos se pueden observar en la figura \ref{FIG:WPP}.

\begin{figure}[Chat Visualizer]{FIG:WPP}{Ejemplos de pantallas de reporte de Chat Visualizer. Fuente: ChatVisualizer.com}
    \image{15cm}{}{wpp}
\end{figure}

\subsection{Limitaciones de las herramientas analizadas}

Estas herramientas existentes, aunque de gran calidad, ninguna ataja el objeto de estudio de este trabajo: La información sensible de los históricos de mensajes.

Las limitaciones que este trabajo pretende subsanar son:
\begin{itemize}
    \item No ofrecer las conclusiones que se pueden obtener de los datos
    \item No es posible analizar más de una conversación en conjunto
    \item Otras aplicaciones de mensajería como Telegram quedan excluidas
    \item No garantizar la privacidad del usuario al procesar los datos en el lado del servidor
\end{itemize}

\newpage
