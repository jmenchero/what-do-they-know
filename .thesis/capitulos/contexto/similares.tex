
Spotify Wrapped:

Desde 2016 Spotify lanza anualmente una campana de marketing que permite a sus usuarios visualizar una compilacion de sus datos de uso, comprandolos con el resto de la comunidad que utiliza la aplicacion. Desde resumenes meramente estadisticos, asi como analisis de emociones o intereses.

Esta camapana ha tenido un gran impacto social y cultural, llegando a formar parte de la cultura pop de las nuevas generaciones, y generando de una forma inconsciente inquietud e interes por los datos a la poblacion general, regularmente ajena y desinteresada de sus datos en la era digital.

Tinder Insights:

Creada en 2019 por Dora Szucs (Software Engineer) y Krisztina Szucs (UX Designer), da acceso a informacion estadistica sobre el uso de la aplicacion Tinder, como la cantidad de mensajes enviados y recibidos, tiempo medio de chat, ...

Ejemplo practico de que el interes por la informacion sobre uno mismo puede ser suficiente para que un gran numero de poblacion comparta sus datos de uso con una aplicacion de terceros para analizarla.

https://tinderinsights.com/contact
https://who.is/whois/tinderinsights.com

Chat Visualizer:

Una herramienta que permite analizar la actividad de un chat de What's App de manera estadistica (72K chats analizados en el momento de la redaccion de esta memoria).

Contras que este trabajo pretende subsanar:
\begin{itemize}
    \item Solo permite analizar una conversacion
    \item Solo analiza informacion de What's App
    \item Procesa los datos en el lado del servidor, no garantizando la privacidad de esa informacion
    \item Solo ofrece un analisis estadistico de actividad
\end{itemize}
