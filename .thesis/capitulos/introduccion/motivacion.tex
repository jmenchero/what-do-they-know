
En la actualidad, como usuarios de un gran número de servicios informáticos gratuitos, generamos una cantidad inmensa de información sobre nosotros mismos todos los días a la que solo tienen acceso las plataformas que nos prestan los servicios que consumimos \cite{DailyData}.

La mayoría de la población, en muchos casos hasta ajena de la definición de metainformación, ni siquiera puede plantearse ser conscientes de las conclusiones que son capaces de sacar sobre nosotros mismos a partir de nuestros datos, lo que conlleva un descuido cada vez más asumido sobre nuestra privacidad.

Este trabajo propone las bases para una plataforma de acceso libre y gratuito, que podrá expandirse gracias a la comunidad opensource, a lo que típicamente resultaría un proyecto interno de análisis de datos para cualquier empresa que tuviera acceso a los mismos para atajar esa falta de conocimiento.
