
En este trabajo presento una aplicación web donde los usuarios pueden analizar sus copias de seguridad de aplicaciones de mensajería instantánea (como Telegram o What's App) y visualizar varios reportes con información que se puede destilar en base a esos históricos.

A lo largo de los siguientes capítulos de la memoria detallo:
\begin{enumerate}
    \item \textbf{Estado del arte}: Se profundiza en el marco legal que permite la realización de este proyecto, el estudio de otros proyectos similares así como un pequeño resumen del contexto y alternativas de las principales tecnologías elegidas (AWS, Redshift y Vue).
    \item \textbf{Análisis de requisitos}: Se recopilan todos los requisitos tanto funcionales como no funcionales identificados, en los que se basan las decisiones tecnológicas detalladas más adelante.
    \item \textbf{Diseño}: Se realiza una propuesta de experiencia de usuario que satisfaga todos los requisitos funcionales indicados previamente, la infraestructura que de soporte a los requisitos no funcionales y una descripción de la planificación de la ejecución a seguir.
    \item \textbf{Implementación}: Profundiza en la solución de código adoptada tanto para la infraestructura como para la plataforma.
    \item \textbf{Pruebas}: Se resumen las pruebas realizadas para garantizar la calidad del proyecto.
    \item \textbf{Conclusiones y trabajo futuro}: Recopila las conclusiones alcanzadas tras el desarrollo de este trabajo y propone una serie de futuros pasos a seguir tras su defensa.
\end{enumerate}
