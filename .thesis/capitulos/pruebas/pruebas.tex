
Como se espera liberar el acceso a la plataforma al público general, resulta fundamental asegurar la funcionalidad de la herramienta. Para ello se han utilizado cinco históricos reales de voluntarios y voluntarias pertenecientes a la Escuela Politécnica Superior de la Universidad Autónoma de Madrid para probar el flujo de datos y generar los reportes. Dichos informes no se comparten para conservar la privacidad de los voluntarios. No obstante también se ha empleado un histórico sintético de apenas 20 mensajes para garantizar que el flujo de datos y los cálculos sean correctos, este sí accesible en el repositorio del proyecto\cite{Repositorio} (en el fichero .tests/synthetic.json). Dichas pruebas se han realizado a lo largo de la implementación tras cada etapa de desarrollo y han resultado en una serie de correcciones que se detallan en la siguiente lista de commits.

\begin{enumerate}
    \item (Feb 1, 2022 - 394b724) \textbf{Add store persistance}: Al actualizar el navegador se perdían los análisis generados.
    \item (Mar 23, 2022 - e43ca44) \textbf{Fix emojis wall limits}: Los emojis se salían en algunos casos de los límites de la pantalla. 
    \item (Mar 23, 2022 - 7d0172f) \textbf{Make active hours data reactive}: El gráfico de horas no respondía a cambios en los datos tras la carga. 
    \item (May 24, 2022 - 2105139) \textbf{Fix redshift naming requirements}: Diferentes atributos de la configuración de Redshift requerían de formatos diferentes.
    \item (Jun 10, 2022 - 70bf47c) \textbf{Refactor how messages are stored and analysed}: La persistencia no funcionaba debido a que se superaban los límites de almacenamiento de los navegadores.
    \item (Jun 10, 2022 - 3692b3d) \textbf{Fix unavailable redshift result by awaiting}: Ciertas consultas no devolvían datos porque se consultaba su resultado antes de que este estuviera disponible.
    \item (Jun 12, 2022 - 0b1f44a) \textbf{Fix visuals from emojis wall}: Los emojis tapaban textos y botonería en algunos casos.
\end{enumerate}

Se ha descartado realizar tests unitarios en las fases de prototipado de la herramienta ya que la estructura era sujeto de cambio constante, pero estos se incluiran en futuras versiones una vez se alcance la estabilidad estructural del código, así como tests end to end para garantizar que la funcionalidad básica esté siempre cubierta.
