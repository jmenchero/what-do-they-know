
% (Extra: en 5 ambitos) https://www.keboola.com/blog/relational-vs-non-relational-database-when-to-use-one-instead-of-the-other#:~:text=Relational%20databases%20are%20best%20for,best%20for%20different%20data%20structures.

\begin{itemize}
    \item Redshift, columnar por ser analitica, poder usar data api y no tener que montar otra capa. posibilidad de escalado.
    \item DC2 porque es gratis [https://aws.amazon.com/es/redshift/pricing/](https://aws.amazon.com/es/redshift/pricing/)
    \item Acceder a redshift primero desde cli y luego desde js sdk [https://docs.aws.amazon.com/cli/latest/userguide/getting-started-install.html](https://docs.aws.amazon.com/cli/latest/userguide/getting-started-install.html)
\end{itemize}

Working with the Amazon Redshift Data API

Before you use the Amazon Redshift Data API, review the following steps:

\begin{enumerate}
    \item Determine if you, as the caller of the Data API, are authorized. 
    \item Determine if you plan to call the Data API with authentication credentials from Secrets Manager or temporary credentials. 
    \item Set up a secret if you use Secrets Manager for authentication credentials. 
    \item Review the considerations and limitations when calling the Data API. 
    \item Call the Data API from the AWS Command Line Interface (AWS CLI), from your own code, or using the query editor in the Amazon Redshift console.
\end{enumerate}
